\documentclass{article}

% Language setting
% Replace `english' with e.g. `spanish' to change the document language
\usepackage[spanish]{babel}

% Set page size and margins
% Replace `letterpaper' with `a4paper' for UK/EU standard size
\usepackage[a4paper,top=2cm,bottom=2cm,left=3cm,right=3cm,marginparwidth=1.75cm]{geometry}

%Parskip and parindent
\setlength{\parindent}{0pt}
\setlength{\parskip}{1.0em}

% Useful packages
\usepackage{amsmath}
\usepackage{graphicx}
\usepackage[colorlinks=true, allcolors=blue]{hyperref}

\begin{document}


\section{Definición del Problema}

Los glaciares constituyen una de las mayores reservas de agua dulce del planeta y cumplen un rol fundamental en la regulación de ecosistemas y la provisión de recursos. En las últimas décadas, se han observado procesos de retroceso glaciar que ponen en evidencia la necesidad de contar un monitoreo constante y sistemático que permita mejorar nuestro entendimiento del comportamiento de estas masas de hielo. La comprensión de su evolución resulta esencial no solo para la conservación de un recurso natural estratégico y la definición de políticas ambientales, sino también es fundamental para la gestión de los recursos hídricos, anticipar posibles impactos en la disponibilidad de agua y para la planificación territorial.

El seguimiento tradicional de glaciares, basado en campañas de campo e inventarios periódicos, presenta limitaciones significativas: requiere una logística compleja, altos costos, y ofrece información de baja frecuencia temporal. En paralelo, la creciente disponibilidad de imágenes satelitales de acceso abierto, provistas por programas internacionales como Landsat y Sentinel, abre la posibilidad de generar metodologías de observación más frecuentes, sistemáticas y escalables. Estas imágenes, combinadas con técnicas de procesamiento digital y herramientas de ciencia de datos, permiten abordar el análisis de la dinámica glaciar.

El interés de este trabajo se centra en explorar el potencial de dichas herramientas para desarrollar un enfoque que contribuya al monitoreo y análisis de glaciares. Si bien se prevé que la atención principal se coloque en glaciares ubicados en el territorio argentino, dada su relevancia en el contexto local, no se descarta la incorporación de casos de otras regiones. De esta manera, se busca aportar una visión comparativa y generar aprendizajes que puedan ser útiles tanto en el ámbito nacional como internacional.

\end{document}
