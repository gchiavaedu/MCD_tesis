\documentclass[10pt]{article}

\usepackage[spanish]{babel}  
\usepackage[utf8]{inputenc} 
\usepackage{graphicx}
\usepackage{subfigure}
\usepackage{amsfonts}
\usepackage{comment}
\usepackage{amssymb,amsmath}
\usepackage{graphicx,psfrag,float}




\voffset-2.5cm 
\hoffset 0.0cm 
\topmargin 2cm
\oddsidemargin-5mm% 
\evensidemargin 0mm
\textwidth=17cm
\textheight=23.2cm 
%\columnsep=0.8cm
\pagestyle{empty}
\renewcommand{\baselinestretch}{1}  
\sloppy  
\widowpenalty100000   
\clubpenalty100000
\parskip 2mm   


\begin{document}
\title{{\bf Plan de Trabajo Tesis de Maestría}\\ Monitoreo de Glaciares utilizando Imágenes Satelitales Ópticas
}
\author{ 
	 \bf{Alumno}: Guido Chiavarini\\
 \ \\
\bf{Tutoras:  Juliana Gambini - Andrea Rey}}
\date{}
\maketitle
	\Large{\textbf{}}
	
	\section{Descripción del problema}



\section{Objetivos específicos}
Se propone realizar las siguientes actividades: 
\begin{enumerate}
	
	\item Realizar el Estado del arte acerca de los diferentes tipos de glaciares existentes y métodos que se utilizan en su estudio.
	\item Descargar una secuencia temporal de imágenes satelitales ópticas que involucren glaciares.
	\item Estudiar indicadores especiales para la detección de nieve y medición de la extención de glaciares. 
	\item Diseñar y entrenar modelos de machine learning para la interpretación de estas imágenes con el objetivo de medir la extención de los glaciares y observar cambios en el tiempo.
	\item Validación de los modelos con metodologías preexistentes.
	\item Evaluación de resultados.
	\item Escritura de la tesis.
	
	
	
	
\end{enumerate}


\section{Cronograma de trabajo}
%max 1 pag.

El Cronograma de Trabajo se muestra en el siguiente cuadro:




\begin{table}[!ht]
	\label{tab:kurskew}
	\begin{center}
		\begin{tabular}{| c | c | c | c | c | c | c | c | c | c | c |c | c |c |c |c |}\cline{1-16}
			\multicolumn{1}{|c|}{Actividad} & \multicolumn{15}{|c|}{Mes}\\ \hline
			\multicolumn{1}{|c|}{ } &	\multicolumn{1}{|c|}{ 1} & 					\multicolumn{1}{|c|}{2} & \multicolumn{1}{|c|}{3} & 		\multicolumn{1}{|c|}{4} & \multicolumn{1}{|c|}{ 5} & 					\multicolumn{1}{|c|}{6} & \multicolumn{1}{|c|}{7} & 		\multicolumn{1}{|c|}{8} & \multicolumn{1}{|c|}{ 9} & 					\multicolumn{1}{|c|}{10} & \multicolumn{1}{|c|}{11} & 		\multicolumn{1}{|c|}{12} & \multicolumn{1}{|c|}{13}& \multicolumn{1}{|c|}{14}& \multicolumn{1}{|c|}{15} \\ \hline
			Actividad 1 &  x& x & x &   &   &  &  & & & & & & & & \\ \hline
			Actividad  2 &  &  &  x&   &   & x &  & & & & & & & & \\ \hline
			Actividad 3  &  &  &  &  x & x& x & x &  &  &  & &   & & & \\ \hline
			Actividad  4 y 5 &  &  &  &   &  &   &   & x& x& x & &  & &  &  \\ \hline
			Actividad  6 &  &  &  &   &  &  &   & & & & x& x & x& x & x\\ \hline
			%Actividad 1&  x& x &  x&  x & x& x & x  & x&x & x& x& x \\ \hline
		\end{tabular}
	\end{center}
\end{table}
\end{document}